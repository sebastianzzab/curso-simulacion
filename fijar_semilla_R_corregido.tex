
\documentclass[12pt,a4paper]{article}
\usepackage[T1]{fontenc}
\usepackage[utf8]{inputenc}
\usepackage[spanish]{babel}
\usepackage{amsmath}
\usepackage{listings}
\usepackage{xcolor}
\usepackage[hidelinks]{hyperref}

\lstdefinestyle{Rstyle}{
    language=R,
    backgroundcolor=\color{gray!10},
    basicstyle=\ttfamily\small,
    breaklines=true,
    frame=single,
    showstringspaces=false,
    keywordstyle=\color{blue},
    commentstyle=\color{gray!70},
    stringstyle=\color{orange!90!black}
}
\lstset{style=Rstyle}

\title{Fijar la Semilla en R: Reproducibilidad en Simulación}
\author{Sebastián Ernesto Zabala Zabala}
\date{Octubre 2025}

\begin{document}

\maketitle

\section*{Introducción}
En simulación y modelado estadístico, la reproducibilidad de los resultados es un aspecto fundamental. En R, esta se garantiza mediante el uso de la función \texttt{set.seed()}, que permite fijar la semilla del generador de números aleatorios.

\section*{Funcionamiento de \texttt{set.seed()}}
La función \texttt{set.seed()} inicializa el generador de números aleatorios, asegurando que la secuencia generada sea la misma cada vez que se ejecute el código. Esto permite reproducir experimentos y garantizar la consistencia de los resultados.

\begin{lstlisting}[language=R]
set.seed(1)
rnorm(5)
# [1] -0.6264538  0.1836433 -0.8356286  1.5952808  0.3295078
\end{lstlisting}

\section*{Ejemplo práctico}
Si ejecutamos el siguiente código sin establecer una semilla, los resultados variarán cada vez:

\begin{lstlisting}[language=R]
rnorm(5)
\end{lstlisting}

Sin embargo, al fijar la semilla, los resultados serán idénticos en cada ejecución:

\begin{lstlisting}[language=R]
set.seed(123)
rnorm(5)
# [1] -0.5604756 -0.2301775  1.5587083  0.0705084  0.1292877
\end{lstlisting}

\section*{Uso avanzado}
También es posible utilizar variables dinámicas como la hora del sistema para generar semillas únicas en cada ejecución:

\begin{lstlisting}[language=R]
set.seed(as.numeric(Sys.time()))
rnorm(5)
\end{lstlisting}

Este enfoque es útil cuando se desea asegurar aleatoriedad sin reproducibilidad.

\section*{Importancia en simulación}
En experimentos de simulación, fijar la semilla es crucial para comparar métodos o verificar resultados. Por ejemplo, al realizar replicaciones Monte Carlo, se recomienda fijar una semilla al inicio de cada iteración o experimento.

\begin{lstlisting}[language=R]
set.seed(100)
n_rep <- 1000
n <- 30
Mediana <- numeric(n_rep)

for (i in 1:n_rep) {
  Mediana[i] <- median(runif(n))
}

mean(Mediana)
\end{lstlisting}

\section*{Conclusión}
El uso de \texttt{set.seed()} es una práctica esencial para cualquier estadístico o científico de datos que trabaje con simulaciones. Garantiza reproducibilidad, transparencia y rigor en el proceso analítico.

\end{document}
