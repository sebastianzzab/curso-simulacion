% Options for packages loaded elsewhere
% Options for packages loaded elsewhere
\PassOptionsToPackage{unicode}{hyperref}
\PassOptionsToPackage{hyphens}{url}
\PassOptionsToPackage{dvipsnames,svgnames,x11names}{xcolor}
%
\documentclass[
  spanish,
  letterpaper,
  DIV=11,
  numbers=noendperiod]{scrartcl}
\usepackage{xcolor}
\usepackage{amsmath,amssymb}
\setcounter{secnumdepth}{-\maxdimen} % remove section numbering
\usepackage{iftex}
\ifPDFTeX
  \usepackage[T1]{fontenc}
  \usepackage[utf8]{inputenc}
  \usepackage{textcomp} % provide euro and other symbols
\else % if luatex or xetex
  \usepackage{unicode-math} % this also loads fontspec
  \defaultfontfeatures{Scale=MatchLowercase}
  \defaultfontfeatures[\rmfamily]{Ligatures=TeX,Scale=1}
\fi
\usepackage{lmodern}
\ifPDFTeX\else
  % xetex/luatex font selection
\fi
% Use upquote if available, for straight quotes in verbatim environments
\IfFileExists{upquote.sty}{\usepackage{upquote}}{}
\IfFileExists{microtype.sty}{% use microtype if available
  \usepackage[]{microtype}
  \UseMicrotypeSet[protrusion]{basicmath} % disable protrusion for tt fonts
}{}
\makeatletter
\@ifundefined{KOMAClassName}{% if non-KOMA class
  \IfFileExists{parskip.sty}{%
    \usepackage{parskip}
  }{% else
    \setlength{\parindent}{0pt}
    \setlength{\parskip}{6pt plus 2pt minus 1pt}}
}{% if KOMA class
  \KOMAoptions{parskip=half}}
\makeatother
% Make \paragraph and \subparagraph free-standing
\makeatletter
\ifx\paragraph\undefined\else
  \let\oldparagraph\paragraph
  \renewcommand{\paragraph}{
    \@ifstar
      \xxxParagraphStar
      \xxxParagraphNoStar
  }
  \newcommand{\xxxParagraphStar}[1]{\oldparagraph*{#1}\mbox{}}
  \newcommand{\xxxParagraphNoStar}[1]{\oldparagraph{#1}\mbox{}}
\fi
\ifx\subparagraph\undefined\else
  \let\oldsubparagraph\subparagraph
  \renewcommand{\subparagraph}{
    \@ifstar
      \xxxSubParagraphStar
      \xxxSubParagraphNoStar
  }
  \newcommand{\xxxSubParagraphStar}[1]{\oldsubparagraph*{#1}\mbox{}}
  \newcommand{\xxxSubParagraphNoStar}[1]{\oldsubparagraph{#1}\mbox{}}
\fi
\makeatother

\usepackage{color}
\usepackage{fancyvrb}
\newcommand{\VerbBar}{|}
\newcommand{\VERB}{\Verb[commandchars=\\\{\}]}
\DefineVerbatimEnvironment{Highlighting}{Verbatim}{commandchars=\\\{\}}
% Add ',fontsize=\small' for more characters per line
\usepackage{framed}
\definecolor{shadecolor}{RGB}{241,243,245}
\newenvironment{Shaded}{\begin{snugshade}}{\end{snugshade}}
\newcommand{\AlertTok}[1]{\textcolor[rgb]{0.68,0.00,0.00}{#1}}
\newcommand{\AnnotationTok}[1]{\textcolor[rgb]{0.37,0.37,0.37}{#1}}
\newcommand{\AttributeTok}[1]{\textcolor[rgb]{0.40,0.45,0.13}{#1}}
\newcommand{\BaseNTok}[1]{\textcolor[rgb]{0.68,0.00,0.00}{#1}}
\newcommand{\BuiltInTok}[1]{\textcolor[rgb]{0.00,0.23,0.31}{#1}}
\newcommand{\CharTok}[1]{\textcolor[rgb]{0.13,0.47,0.30}{#1}}
\newcommand{\CommentTok}[1]{\textcolor[rgb]{0.37,0.37,0.37}{#1}}
\newcommand{\CommentVarTok}[1]{\textcolor[rgb]{0.37,0.37,0.37}{\textit{#1}}}
\newcommand{\ConstantTok}[1]{\textcolor[rgb]{0.56,0.35,0.01}{#1}}
\newcommand{\ControlFlowTok}[1]{\textcolor[rgb]{0.00,0.23,0.31}{\textbf{#1}}}
\newcommand{\DataTypeTok}[1]{\textcolor[rgb]{0.68,0.00,0.00}{#1}}
\newcommand{\DecValTok}[1]{\textcolor[rgb]{0.68,0.00,0.00}{#1}}
\newcommand{\DocumentationTok}[1]{\textcolor[rgb]{0.37,0.37,0.37}{\textit{#1}}}
\newcommand{\ErrorTok}[1]{\textcolor[rgb]{0.68,0.00,0.00}{#1}}
\newcommand{\ExtensionTok}[1]{\textcolor[rgb]{0.00,0.23,0.31}{#1}}
\newcommand{\FloatTok}[1]{\textcolor[rgb]{0.68,0.00,0.00}{#1}}
\newcommand{\FunctionTok}[1]{\textcolor[rgb]{0.28,0.35,0.67}{#1}}
\newcommand{\ImportTok}[1]{\textcolor[rgb]{0.00,0.46,0.62}{#1}}
\newcommand{\InformationTok}[1]{\textcolor[rgb]{0.37,0.37,0.37}{#1}}
\newcommand{\KeywordTok}[1]{\textcolor[rgb]{0.00,0.23,0.31}{\textbf{#1}}}
\newcommand{\NormalTok}[1]{\textcolor[rgb]{0.00,0.23,0.31}{#1}}
\newcommand{\OperatorTok}[1]{\textcolor[rgb]{0.37,0.37,0.37}{#1}}
\newcommand{\OtherTok}[1]{\textcolor[rgb]{0.00,0.23,0.31}{#1}}
\newcommand{\PreprocessorTok}[1]{\textcolor[rgb]{0.68,0.00,0.00}{#1}}
\newcommand{\RegionMarkerTok}[1]{\textcolor[rgb]{0.00,0.23,0.31}{#1}}
\newcommand{\SpecialCharTok}[1]{\textcolor[rgb]{0.37,0.37,0.37}{#1}}
\newcommand{\SpecialStringTok}[1]{\textcolor[rgb]{0.13,0.47,0.30}{#1}}
\newcommand{\StringTok}[1]{\textcolor[rgb]{0.13,0.47,0.30}{#1}}
\newcommand{\VariableTok}[1]{\textcolor[rgb]{0.07,0.07,0.07}{#1}}
\newcommand{\VerbatimStringTok}[1]{\textcolor[rgb]{0.13,0.47,0.30}{#1}}
\newcommand{\WarningTok}[1]{\textcolor[rgb]{0.37,0.37,0.37}{\textit{#1}}}

\usepackage{longtable,booktabs,array}
\usepackage{calc} % for calculating minipage widths
% Correct order of tables after \paragraph or \subparagraph
\usepackage{etoolbox}
\makeatletter
\patchcmd\longtable{\par}{\if@noskipsec\mbox{}\fi\par}{}{}
\makeatother
% Allow footnotes in longtable head/foot
\IfFileExists{footnotehyper.sty}{\usepackage{footnotehyper}}{\usepackage{footnote}}
\makesavenoteenv{longtable}
\usepackage{graphicx}
\makeatletter
\newsavebox\pandoc@box
\newcommand*\pandocbounded[1]{% scales image to fit in text height/width
  \sbox\pandoc@box{#1}%
  \Gscale@div\@tempa{\textheight}{\dimexpr\ht\pandoc@box+\dp\pandoc@box\relax}%
  \Gscale@div\@tempb{\linewidth}{\wd\pandoc@box}%
  \ifdim\@tempb\p@<\@tempa\p@\let\@tempa\@tempb\fi% select the smaller of both
  \ifdim\@tempa\p@<\p@\scalebox{\@tempa}{\usebox\pandoc@box}%
  \else\usebox{\pandoc@box}%
  \fi%
}
% Set default figure placement to htbp
\def\fps@figure{htbp}
\makeatother



\ifLuaTeX
\usepackage[bidi=basic]{babel}
\else
\usepackage[bidi=default]{babel}
\fi
% get rid of language-specific shorthands (see #6817):
\let\LanguageShortHands\languageshorthands
\def\languageshorthands#1{}


\setlength{\emergencystretch}{3em} % prevent overfull lines

\providecommand{\tightlist}{%
  \setlength{\itemsep}{0pt}\setlength{\parskip}{0pt}}



 


\KOMAoption{captions}{tableheading}
\makeatletter
\@ifpackageloaded{tcolorbox}{}{\usepackage[skins,breakable]{tcolorbox}}
\@ifpackageloaded{fontawesome5}{}{\usepackage{fontawesome5}}
\definecolor{quarto-callout-color}{HTML}{909090}
\definecolor{quarto-callout-note-color}{HTML}{0758E5}
\definecolor{quarto-callout-important-color}{HTML}{CC1914}
\definecolor{quarto-callout-warning-color}{HTML}{EB9113}
\definecolor{quarto-callout-tip-color}{HTML}{00A047}
\definecolor{quarto-callout-caution-color}{HTML}{FC5300}
\definecolor{quarto-callout-color-frame}{HTML}{acacac}
\definecolor{quarto-callout-note-color-frame}{HTML}{4582ec}
\definecolor{quarto-callout-important-color-frame}{HTML}{d9534f}
\definecolor{quarto-callout-warning-color-frame}{HTML}{f0ad4e}
\definecolor{quarto-callout-tip-color-frame}{HTML}{02b875}
\definecolor{quarto-callout-caution-color-frame}{HTML}{fd7e14}
\makeatother
\makeatletter
\@ifpackageloaded{caption}{}{\usepackage{caption}}
\AtBeginDocument{%
\ifdefined\contentsname
  \renewcommand*\contentsname{Tabla de contenidos}
\else
  \newcommand\contentsname{Tabla de contenidos}
\fi
\ifdefined\listfigurename
  \renewcommand*\listfigurename{Listado de Figuras}
\else
  \newcommand\listfigurename{Listado de Figuras}
\fi
\ifdefined\listtablename
  \renewcommand*\listtablename{Listado de Tablas}
\else
  \newcommand\listtablename{Listado de Tablas}
\fi
\ifdefined\figurename
  \renewcommand*\figurename{Figura}
\else
  \newcommand\figurename{Figura}
\fi
\ifdefined\tablename
  \renewcommand*\tablename{Tabla}
\else
  \newcommand\tablename{Tabla}
\fi
}
\@ifpackageloaded{float}{}{\usepackage{float}}
\floatstyle{ruled}
\@ifundefined{c@chapter}{\newfloat{codelisting}{h}{lop}}{\newfloat{codelisting}{h}{lop}[chapter]}
\floatname{codelisting}{Listado}
\newcommand*\listoflistings{\listof{codelisting}{Listado de Listados}}
\makeatother
\makeatletter
\makeatother
\makeatletter
\@ifpackageloaded{caption}{}{\usepackage{caption}}
\@ifpackageloaded{subcaption}{}{\usepackage{subcaption}}
\makeatother
\usepackage{bookmark}
\IfFileExists{xurl.sty}{\usepackage{xurl}}{} % add URL line breaks if available
\urlstyle{same}
\hypersetup{
  pdftitle={Generación de números pseudoaleatorios mediante el método de los cuadrados medios},
  pdfauthor={Luis Ríos; Sebastian Zabala},
  pdflang={es},
  colorlinks=true,
  linkcolor={blue},
  filecolor={Maroon},
  citecolor={Blue},
  urlcolor={Blue},
  pdfcreator={LaTeX via pandoc}}


\title{Generación de números pseudoaleatorios mediante el método de los
cuadrados medios}
\author{Luis Ríos \and Sebastian Zabala}
\date{2025-11-04}
\begin{document}
\maketitle


\section{Importancia de los números
pseudoaleatorios}\label{importancia-de-los-nuxfameros-pseudoaleatorios}

Los números pseudoaleatorios son fundamentales en diversas áreas de la
informática y las matemáticas, incluyendo simulaciones, criptografía,
juegos y algoritmos de optimización. A diferencia de los números
verdaderamente aleatorios, que se generan a partir de fenómenos físicos
impredecibles, los números pseudoaleatorios se generan mediante
algoritmos deterministas que producen secuencias de números que imitan
las propiedades estadísticas de los números aleatorios.

\section{Métodos comunes para generar números
pseudoaleatorios}\label{muxe9todos-comunes-para-generar-nuxfameros-pseudoaleatorios}

Existen varios métodos para generar números pseudoaleatorios, entre los
cuales destacan:

\begin{enumerate}
\def\labelenumi{\arabic{enumi}.}
\tightlist
\item
  \textbf{Generadores Congruenciales Lineales (LCG)}: Este es uno de los
  métodos más simples y ampliamente utilizados. Utiliza una fórmula
  matemática para generar una secuencia de números basada en una semilla
  inicial. La fórmula general es: \[
  \begin{equation*}
  X_{n+1} = (aX_n + c) \mod m
  \end{equation*}
  \] donde \((X)\) es el número pseudoaleatorio, (\(a\)), (\(c\)) y
  (\(m\)) son constantes, y (\(X_0\)) es la semilla inicial.
\item
  \textbf{Generadores de Mersenne Twister}: Este es un generador de
  números pseudoaleatorios más avanzado que ofrece una mayor calidad y
  un período más largo que los LCG. Es ampliamente utilizado en
  aplicaciones científicas y de simulación.
\item
  \textbf{Generadores basados en hardware}: Algunos sistemas utilizan
  fuentes de entropía física, como ruido térmico o movimientos del
  mouse, para generar números aleatorios. Aunque estos no son
  estrictamente pseudoaleatorios, a menudo se combinan con métodos
  algorítmicos para mejorar la calidad de los números generados.
\end{enumerate}

\section{Método de los Cuadrados
Medios}\label{muxe9todo-de-los-cuadrados-medios}

Uno de los métodos clásicos para generar números pseudoaleatorios es el
\textbf{método de los cuadrados medios}. Siendo uno de los primeros
métodos propuestos históricamente para la generación de números
pseudoaleatorios, introducido por John von Neumann en 1946. A pesar de
que hoy en día se considera un método con limitaciones importantes en
cuanto a la calidad de las secuencias que produce, resulta útil para
fines didácticos, pues permite comprender de forma clara cómo un
algoritmo determinista puede generar una secuencia de valores a partir
de una semilla inicial.

\subsection{Pasos para aplicar el método de los cuadrados
medios}\label{pasos-para-aplicar-el-muxe9todo-de-los-cuadrados-medios}

Este método implica los siguientes pasos: i. Se escoge un número de
cuatro dígitos \(x_0\) (semilla). ii. Se eleva al cuadrado (\(x_0^2\)) y
se toman los cuatro dígitos centrales (\(x_1\)). iii. Se genera el el
número pseudo-aleatorio como \[
\begin{equation*}
u_1 = \frac{x_1}{10^4}
\end{equation*}
\] iv. Volver al paso ii y repetir el proceso. Para obtener los \(k\)
(número par) dígitos centrales de \(x_i^2\) se puede utilizar que: \[
\begin{equation*}
x_{i+1} = \left\lfloor \left( x_i^2 - \left\lfloor \frac{x_i^2}{10^{\left(2k - \frac{k}{2}\right)}} \right\rfloor 10^{\left(2k - \frac{k}{2}\right)} \right) / 10^{\frac{k}{2}} \right\rfloor
\end{equation*}
\]

\subsection{\texorpdfstring{Implementación del método de los cuadrados
medios en
{R}}{Implementación del método de los cuadrados medios en R}}\label{implementaciuxf3n-del-muxe9todo-de-los-cuadrados-medios-en-r}

\paragraph{\texorpdfstring{Ejemplo con semilla 1234 y 100 valores
utilizando el paquete
\texttt{simres}}{Ejemplo con semilla 1234 y 100 valores utilizando el paquete simres}}\label{ejemplo-con-semilla-1234-y-100-valores-utilizando-el-paquete-simres}

Para ejemplificar el funcionamiento del método de los cuadrados medios,
se generarán 100 números pseudoaleatorios utilizando una semilla inicial
de 1234 (\(x_0=1234\)) con \(k=4\) dígitos centrales. Todo esto usando
la función \texttt{simres::rvng()} del paquete \texttt{simres}.

\begin{Shaded}
\begin{Highlighting}[]
\FunctionTok{library}\NormalTok{(simres)}
\NormalTok{simres}\SpecialCharTok{::}\FunctionTok{rvng}\NormalTok{(}\DecValTok{100}\NormalTok{, }\DecValTok{1234}\NormalTok{) }
\end{Highlighting}
\end{Shaded}

\begin{verbatim}
  [1] 0.5227 0.3215 0.3362 0.3030 0.1809 0.2724 0.4201 0.6484 0.0422 0.1780
 [11] 0.1684 0.8358 0.8561 0.2907 0.4506 0.3040 0.2416 0.8370 0.0569 0.3237
 [21] 0.4781 0.8579 0.5992 0.9040 0.7216 0.0706 0.4984 0.8402 0.5936 0.2360
 [31] 0.5696 0.4444 0.7491 0.1150 0.3225 0.4006 0.0480 0.2304 0.3084 0.5110
 [41] 0.1121 0.2566 0.5843 0.1406 0.9768 0.4138 0.1230 0.5129 0.3066 0.4003
 [51] 0.0240 0.0576 0.3317 0.0024 0.0005 0.0000 0.0000 0.0000 0.0000 0.0000
 [61] 0.0000 0.0000 0.0000 0.0000 0.0000 0.0000 0.0000 0.0000 0.0000 0.0000
 [71] 0.0000 0.0000 0.0000 0.0000 0.0000 0.0000 0.0000 0.0000 0.0000 0.0000
 [81] 0.0000 0.0000 0.0000 0.0000 0.0000 0.0000 0.0000 0.0000 0.0000 0.0000
 [91] 0.0000 0.0000 0.0000 0.0000 0.0000 0.0000 0.0000 0.0000 0.0000 0.0000
\end{verbatim}

Como se puede observar, llegado un punto de la secuencia, los números
generados son todos ceros (0). Relevando asi uno de los principales
problemas del método de cuadrados medios, \textbf{la absorción}. El
problema es que si en algún paso se obtiene un número pequeño (pocos
dígitos), al rellenar con ceros y tomar los dígitos centrales se pueden
sacar muchos ceros, llegando asi a cero (0). Y si \(x_n=0\), entonces
\(x_{n+1}=0\) para siempre.

\begin{tcolorbox}[enhanced jigsaw, coltitle=black, colback=white, breakable, colbacktitle=quarto-callout-warning-color!10!white, toprule=.15mm, arc=.35mm, opacityback=0, bottomtitle=1mm, bottomrule=.15mm, opacitybacktitle=0.6, left=2mm, colframe=quarto-callout-warning-color-frame, leftrule=.75mm, rightrule=.15mm, titlerule=0mm, toptitle=1mm, title=\textcolor{quarto-callout-warning-color}{\faExclamationTriangle}\hspace{0.5em}{Advertencia}]

\textbf{Advertencia:} El método de los cuadrados medios puede degenerar
rápidamente al valor 0.

\end{tcolorbox}

A continuación se muestran las operaciones específicas que provocan la
caída de la secuencia hacia el cero. Consideremos el punto de la
secuencia en el que comienzan a aparecer valores pequeños, lo cual
desencadena la absorción:

\[
x_{53} = 3317
\]

\textbf{1. Primer paso}

\[
x_{53}^2 = 3317^2 = 11002489
\]

Tomando los dígitos centrales:

\[
11002489 \Rightarrow \boxed{0024}
\]

\[
x_{54} = 0024 = 24
\]

\textbf{2. Segundo paso}

\[
x_{54}^2 = 24^2 = 576
\]

Tomando los dígitos centrales:

\[
00000576 \Rightarrow \boxed{0005}
\]

\[
x_{55} = 0005 = 5
\]

\textbf{3. Tercer paso}

\[
x_{55}^2 = 5^2 = 25
\]

Tomando los dígitos centrales:

\[
00000025 \Rightarrow \boxed{0000}
\]

\[
x_{56} = 0000 = 0
\]

A partir de aquí, la secuencia queda atrapada en un \textbf{estado
absorbente}:

\[
x_{57} = 0 \Rightarrow x_{58} = 0 \Rightarrow x_{59} = 0 \Rightarrow \dots
\]

Lo que confirma que, una vez que el método alcanza (0), no es posible
recuperar valores distintos de cero, perdiéndose completamente la
utilidad del generador.

\paragraph{Ejemplo con semilla 1234 y 100 valores utilizando función
propia}\label{ejemplo-con-semilla-1234-y-100-valores-utilizando-funciuxf3n-propia}

A continuación, se presenta una implementación propia del método de los
cuadrados medios en R para generar 100 números pseudoaleatorios a partir
de una semilla inicial de 1234.

\begin{Shaded}
\begin{Highlighting}[]
\NormalTok{cuadrados\_medios }\OtherTok{\textless{}{-}} \ControlFlowTok{function}\NormalTok{(n,}
                             \AttributeTok{seed =} \FunctionTok{as.numeric}\NormalTok{(}\FunctionTok{Sys.time}\NormalTok{()),}
                             \AttributeTok{k =} \DecValTok{4}\NormalTok{,}
                             \AttributeTok{save\_seed =} \ConstantTok{TRUE}\NormalTok{,}
                             \AttributeTok{warn =} \ConstantTok{TRUE}\NormalTok{) \{}
  \CommentTok{\# n: cantidad de números a generar}
  \CommentTok{\# seed: semilla inicial (entero o numérico). Se truncará y reducirá modulo 10\^{}k.}
  \CommentTok{\# k: número de dígitos centrales (debe ser par)}
  \CommentTok{\# save\_seed: si TRUE guarda .rng en globalenv() (semilla final y parámetros)}
  \CommentTok{\# warn: si TRUE muestra advertencias sobre precisión}
  
  \ControlFlowTok{if}\NormalTok{ (}\SpecialCharTok{!}\FunctionTok{is.numeric}\NormalTok{(n) }\SpecialCharTok{||}\NormalTok{ n }\SpecialCharTok{\textless{}=} \DecValTok{0}\NormalTok{) }\FunctionTok{stop}\NormalTok{(}\StringTok{"n debe ser un entero positivo"}\NormalTok{)}
  \ControlFlowTok{if}\NormalTok{ ((k }\SpecialCharTok{\%\%} \DecValTok{2}\NormalTok{) }\SpecialCharTok{!=} \DecValTok{0}\NormalTok{) }\FunctionTok{stop}\NormalTok{(}\StringTok{"k debe ser par"}\NormalTok{)}
  \ControlFlowTok{if}\NormalTok{ (}\SpecialCharTok{!}\FunctionTok{is.numeric}\NormalTok{(seed)) seed }\OtherTok{\textless{}{-}} \FunctionTok{as.numeric}\NormalTok{(seed)}
  
  \CommentTok{\# normalizar semilla: usar solo sus últimos k dígitos}
\NormalTok{  seed }\OtherTok{\textless{}{-}} \FunctionTok{floor}\NormalTok{(seed)}
\NormalTok{  seed }\OtherTok{\textless{}{-}}\NormalTok{ seed }\SpecialCharTok{\%\%}\NormalTok{ (}\DecValTok{10}\SpecialCharTok{\^{}}\NormalTok{k)}
  
  \CommentTok{\# parámetros para extracción por módulo (idénticos al fragmento de simres)}
\NormalTok{  halfk }\OtherTok{\textless{}{-}}\NormalTok{ k }\SpecialCharTok{/} \DecValTok{2}
\NormalTok{  aux   }\OtherTok{\textless{}{-}} \DecValTok{10}\SpecialCharTok{\^{}}\NormalTok{(}\DecValTok{2}\SpecialCharTok{*}\NormalTok{k }\SpecialCharTok{{-}}\NormalTok{ halfk)  }\CommentTok{\# = 10\^{}(2k {-} k/2)  (entero si k par)}
\NormalTok{  aux2  }\OtherTok{\textless{}{-}} \DecValTok{10}\SpecialCharTok{\^{}}\NormalTok{halfk          }\CommentTok{\# = 10\^{}(k/2)}
  
  \CommentTok{\# chequeo de precisión: si z = seed\^{}2 puede exceder la precisión entera segura}
  \CommentTok{\# doble precisión en R mantiene enteros exactos hasta \textasciitilde{}2\^{}53 (\textasciitilde{}9e15).}
  \ControlFlowTok{if}\NormalTok{ (warn) \{}
\NormalTok{    max\_safe }\OtherTok{\textless{}{-}} \DecValTok{2}\SpecialCharTok{\^{}}\DecValTok{53}
    \CommentTok{\# valor máximo de z aproximado: (10\^{}k {-} 1)\^{}2 ≈ 10\^{}(2k)}
\NormalTok{    approx\_z\_max }\OtherTok{\textless{}{-}}\NormalTok{ (}\DecValTok{10}\SpecialCharTok{\^{}}\NormalTok{k)}
    \ControlFlowTok{if}\NormalTok{ (}\DecValTok{10}\SpecialCharTok{\^{}}\NormalTok{(}\DecValTok{2}\SpecialCharTok{*}\NormalTok{k) }\SpecialCharTok{\textgreater{}}\NormalTok{ max\_safe) \{}
      \FunctionTok{warning}\NormalTok{(}\FunctionTok{sprintf}\NormalTok{(}\StringTok{"Para k = \%d la extracción podría perder precisión en z = x\^{}2 (recomendado k \textless{}= 8)."}\NormalTok{, k))}
\NormalTok{    \}}
\NormalTok{  \}}
  
\NormalTok{  u }\OtherTok{\textless{}{-}} \FunctionTok{numeric}\NormalTok{(n)}
  
  \ControlFlowTok{for}\NormalTok{ (i }\ControlFlowTok{in} \FunctionTok{seq\_len}\NormalTok{(n)) \{}
\NormalTok{    z }\OtherTok{\textless{}{-}}\NormalTok{ seed }\SpecialCharTok{*}\NormalTok{ seed}
    \CommentTok{\# implementación equivalente a:}
    \CommentTok{\# seed \textless{}{-} trunc((z {-} trunc(z/aux)*aux)/aux2)}
    \CommentTok{\# pero usando operaciones mod y floor:}
    \CommentTok{\# extraer (z mod aux) then dividir por aux2}
\NormalTok{    seed }\OtherTok{\textless{}{-}} \FunctionTok{floor}\NormalTok{((z }\SpecialCharTok{\%\%}\NormalTok{ aux) }\SpecialCharTok{/}\NormalTok{ aux2)}
\NormalTok{    u[i] }\OtherTok{\textless{}{-}}\NormalTok{ seed }\SpecialCharTok{/}\NormalTok{ (}\DecValTok{10}\SpecialCharTok{\^{}}\NormalTok{k)}
\NormalTok{  \}}
  
  \CommentTok{\# guardar semilla final y parámetros en el entorno global (similar a simres)}
  \ControlFlowTok{if}\NormalTok{ (}\FunctionTok{isTRUE}\NormalTok{(save\_seed)) \{}
    \FunctionTok{assign}\NormalTok{(}\StringTok{".rng"}\NormalTok{,}
           \FunctionTok{list}\NormalTok{(}\AttributeTok{seed =}\NormalTok{ seed, }\AttributeTok{type =} \StringTok{"vm"}\NormalTok{, }\AttributeTok{parameters =} \FunctionTok{list}\NormalTok{(}\AttributeTok{k =}\NormalTok{ k)),}
           \AttributeTok{envir =} \FunctionTok{globalenv}\NormalTok{())}
\NormalTok{  \}}
  
  \FunctionTok{return}\NormalTok{(u)}
\NormalTok{\}}

\CommentTok{\# Generar 100 números pseudoaleatorios con semilla 1234}
\NormalTok{numeros\_pseudoaleatorios }\OtherTok{\textless{}{-}} \FunctionTok{cuadrados\_medios}\NormalTok{(}\DecValTok{100}\NormalTok{, }\DecValTok{1234}\NormalTok{, }\AttributeTok{k =} \DecValTok{4}\NormalTok{)}
\NormalTok{numeros\_pseudoaleatorios}
\end{Highlighting}
\end{Shaded}

\begin{verbatim}
  [1] 0.5227 0.3215 0.3362 0.3030 0.1809 0.2724 0.4201 0.6484 0.0422 0.1780
 [11] 0.1684 0.8358 0.8561 0.2907 0.4506 0.3040 0.2416 0.8370 0.0569 0.3237
 [21] 0.4781 0.8579 0.5992 0.9040 0.7216 0.0706 0.4984 0.8402 0.5936 0.2360
 [31] 0.5696 0.4444 0.7491 0.1150 0.3225 0.4006 0.0480 0.2304 0.3084 0.5110
 [41] 0.1121 0.2566 0.5843 0.1406 0.9768 0.4138 0.1230 0.5129 0.3066 0.4003
 [51] 0.0240 0.0576 0.3317 0.0024 0.0005 0.0000 0.0000 0.0000 0.0000 0.0000
 [61] 0.0000 0.0000 0.0000 0.0000 0.0000 0.0000 0.0000 0.0000 0.0000 0.0000
 [71] 0.0000 0.0000 0.0000 0.0000 0.0000 0.0000 0.0000 0.0000 0.0000 0.0000
 [81] 0.0000 0.0000 0.0000 0.0000 0.0000 0.0000 0.0000 0.0000 0.0000 0.0000
 [91] 0.0000 0.0000 0.0000 0.0000 0.0000 0.0000 0.0000 0.0000 0.0000 0.0000
\end{verbatim}

Del mismo modo, se puede observar que la secuencia generada por esta
función también cae en el estado absorbente de cero (0) después de
varios pasos, confirmando nuevamente la limitación del método de los
cuadrados medios.

\subsection{Análisis de la calidad del generador por el método de los
cuadrados medios para una
secuencia}\label{anuxe1lisis-de-la-calidad-del-generador-por-el-muxe9todo-de-los-cuadrados-medios-para-una-secuencia}

Para evaluar la calidad del generador de números pseudoaleatorios basado
en el método de los cuadrados medios, se pueden utilizar diversas
pruebas estadísticas. A continuación, se presentan algunas de las
pruebas más comunes:

\begin{longtable}[]{@{}ll@{}}
\toprule\noalign{}
Propósito del Análisis & Prueba o Gráfico Utilizado \\
\midrule\noalign{}
\endhead
\bottomrule\noalign{}
\endlastfoot
\textbf{Uniformidad} & Prueba de Chi-cuadrado \\
& Prueba de Kolmogorov-Smirnov \\
& Histograma \\
\textbf{Independencia} & Gráfico Secuencial \\
& Gráfico de Dispersión Retardado \\
& Correlograma (Función de Autocorrelación) \\
& Prueba de Ljung-Box \\
\end{longtable}

\subsection{\texorpdfstring{Implementación de pruebas estadísticas en
{R}}{Implementación de pruebas estadísticas en R}}\label{implementaciuxf3n-de-pruebas-estaduxedsticas-en-r}

Como se pudo observar, el ejemplo anterior con el método de los
cuadrados medios mostró una clara limitación al caer en un estado
absorbente. Por este motivo, se utilizará otra semilla inicial (\(x_0\))
y un valor de \(k\) mayor a 4, esto garantizará que por lo menos podamos
generar 100 números pseudoaleatorios. Si bien esto no soluciona el
problema de la absorción, si que retrasa su aparición. Por último, se
implementarán las pruebas estadísticas mencionadas en {R} para evaluar
la calidad del generador.

A continuación, se generarán 100 números pseudoaleatorios utilizando el
paquete \texttt{simres}, una semilla inicial de 123456 y \(k=6\) dígitos
centrales.

\begin{Shaded}
\begin{Highlighting}[]
\NormalTok{numeros\_generados }\OtherTok{\textless{}{-}}\NormalTok{ simres}\SpecialCharTok{::}\FunctionTok{rvng}\NormalTok{(}\DecValTok{100}\NormalTok{, }\DecValTok{123456}\NormalTok{, }\AttributeTok{k=}\DecValTok{6}\NormalTok{) }
\NormalTok{numeros\_generados}
\end{Highlighting}
\end{Shaded}

\begin{verbatim}
  [1] 0.241383 0.265752 0.624125 0.532015 0.039960 0.596801 0.171433 0.389273
  [9] 0.533468 0.588107 0.869843 0.626844 0.933400 0.235560 0.488513 0.644951
 [17] 0.961792 0.043851 0.922910 0.762868 0.967585 0.220732 0.722615 0.172438
 [25] 0.734863 0.023628 0.558282 0.678791 0.757221 0.383642 0.181184 0.827641
 [33] 0.989624 0.355661 0.494746 0.773604 0.463148 0.506069 0.105832 0.200412
 [41] 0.164969 0.214770 0.126152 0.914327 0.993862 0.761675 0.148805 0.142928
 [49] 0.428413 0.537698 0.119139 0.194101 0.675198 0.892339 0.268890 0.301832
 [57] 0.102556 0.517733 0.047459 0.252356 0.683550 0.240602 0.889322 0.893619
 [65] 0.554917 0.932876 0.257631 0.373732 0.675607 0.444818 0.863053 0.860480
 [73] 0.425830 0.331188 0.685491 0.897911 0.244163 0.615570 0.926424 0.261427
 [81] 0.344076 0.388293 0.771453 0.139731 0.524752 0.364661 0.977644 0.787790
 [89] 0.613084 0.871991 0.368304 0.647836 0.691482 0.147356 0.713790 0.496164
 [97] 0.178714 0.938693 0.144548 0.894124
\end{verbatim}

Como se puede observar, no se tienen números pseudoaleatorios iguales a
cero (0) en esta secuencia generada con una semilla mayor y \(k=6\)
dígitos centrales.

Estos mismos resultados se obtienen utilizando una función propia
implementada en R:

\begin{Shaded}
\begin{Highlighting}[]
\CommentTok{\# Generar 100 números pseudoaleatorios con semilla 1234}
\NormalTok{numeros\_pseudoaleatorios2 }\OtherTok{\textless{}{-}} \FunctionTok{cuadrados\_medios}\NormalTok{(}\DecValTok{100}\NormalTok{, }\DecValTok{123456}\NormalTok{, }\AttributeTok{k =} \DecValTok{6}\NormalTok{)}
\NormalTok{numeros\_pseudoaleatorios2}
\end{Highlighting}
\end{Shaded}

\begin{verbatim}
  [1] 0.241383 0.265752 0.624125 0.532015 0.039960 0.596801 0.171433 0.389273
  [9] 0.533468 0.588107 0.869843 0.626844 0.933400 0.235560 0.488513 0.644951
 [17] 0.961792 0.043851 0.922910 0.762868 0.967585 0.220732 0.722615 0.172438
 [25] 0.734863 0.023628 0.558282 0.678791 0.757221 0.383642 0.181184 0.827641
 [33] 0.989624 0.355661 0.494746 0.773604 0.463148 0.506069 0.105832 0.200412
 [41] 0.164969 0.214770 0.126152 0.914327 0.993862 0.761675 0.148805 0.142928
 [49] 0.428413 0.537698 0.119139 0.194101 0.675198 0.892339 0.268890 0.301832
 [57] 0.102556 0.517733 0.047459 0.252356 0.683550 0.240602 0.889322 0.893619
 [65] 0.554917 0.932876 0.257631 0.373732 0.675607 0.444818 0.863053 0.860480
 [73] 0.425830 0.331188 0.685491 0.897911 0.244163 0.615570 0.926424 0.261427
 [81] 0.344076 0.388293 0.771453 0.139731 0.524752 0.364661 0.977644 0.787790
 [89] 0.613084 0.871991 0.368304 0.647836 0.691482 0.147356 0.713790 0.496164
 [97] 0.178714 0.938693 0.144548 0.894124
\end{verbatim}

Como se puede observar, la secuencia generada por esta función propia
tampoco contiene ceros (0) en los primeros 100 números generados.

A continuación, se aplicarán las pruebas de uniformidad e independencia.

\subsubsection{Análisis de
Uniformidad}\label{anuxe1lisis-de-uniformidad}

La prueba de uniformidad evalúa si los números generados se distribuyen
de manera uniforme en el intervalo {[}0, 1).

\paragraph{Prueba de Chi-cuadrado}\label{prueba-de-chi-cuadrado}

Esta prueba divide el intervalo {[}0, 1) en subintervalos y compara las
frecuencias observadas de números en cada uno con las frecuencias
esperadas. La función \texttt{simres::freq.test()} es una implementación
directa de este contraste para una distribución uniforme.

\begin{Shaded}
\begin{Highlighting}[]
\CommentTok{\# Aplicamos el test de Chi{-}cuadrado}
\NormalTok{simres}\SpecialCharTok{::}\FunctionTok{freq.test}\NormalTok{(numeros\_generados, }\AttributeTok{nclass =} \DecValTok{10}\NormalTok{)}
\end{Highlighting}
\end{Shaded}

\begin{verbatim}

    Chi-squared test for given probabilities

data:  numeros_generados
X-squared = 8, df = 9, p-value = 0.5341
\end{verbatim}

El resultado de la prueba de Chi-cuadrado indica si se puede rechazar la
hipótesis nula de que los números generados siguen una distribución
uniforme. Un valor p alto sugiere que no hay evidencia suficiente para
rechazar la hipótesis nula, indicando una buena uniformidad en la
distribución de los números generados.

\paragraph{Prueba de
Kolmogorov-Smirnov}\label{prueba-de-kolmogorov-smirnov}

La prueba de Kolmogorov-Smirnov compara la distribución empírica de los
números generados con la distribución teórica uniforme. La función
\texttt{ks.test()} en R se utiliza para realizar esta prueba.

\begin{Shaded}
\begin{Highlighting}[]
\CommentTok{\# Aplicamos el test de Kolmogorov{-}Smirnov}
\FunctionTok{ks.test}\NormalTok{(numeros\_generados, }\StringTok{"punif"}\NormalTok{, }\DecValTok{0}\NormalTok{, }\DecValTok{1}\NormalTok{)}
\end{Highlighting}
\end{Shaded}

\begin{verbatim}

    Asymptotic one-sample Kolmogorov-Smirnov test

data:  numeros_generados
D = 0.062556, p-value = 0.8287
alternative hypothesis: two-sided
\end{verbatim}

El resultado de la prueba de Kolmogorov-Smirnov proporciona un valor p
que indica si se puede rechazar la hipótesis nula de que los números
generados siguen una distribución uniforme. Un valor p alto sugiere que
no hay evidencia suficiente para rechazar la hipótesis nula, indicando
una buena uniformidad en la distribución de los números generados.

\subsubsection{Análisis Gráfico:
Histograma}\label{anuxe1lisis-gruxe1fico-histograma}

Un histograma es una representación gráfica que muestra la distribución
de los números generados. Si los números son uniformemente distribuidos,
el histograma debería mostrar barras de alturas similares en todo el
rango.

\begin{Shaded}
\begin{Highlighting}[]
\CommentTok{\# Histograma de la secuencia generada}
\FunctionTok{hist}\NormalTok{(numeros\_generados, }\AttributeTok{freq =} \ConstantTok{FALSE}\NormalTok{, }\AttributeTok{main =} \StringTok{"Histograma de Números Pseudoaleatorios"}\NormalTok{,}
     \AttributeTok{xlab =} \StringTok{"Valor"}\NormalTok{, }\AttributeTok{ylab =} \StringTok{"Densidad"}\NormalTok{, }\AttributeTok{breaks =} \DecValTok{10}\NormalTok{)}
\FunctionTok{abline}\NormalTok{(}\AttributeTok{h =} \DecValTok{1}\NormalTok{, }\AttributeTok{col =} \StringTok{"red"}\NormalTok{, }\AttributeTok{lwd =} \DecValTok{2}\NormalTok{) }\CommentTok{\# Línea de la densidad uniforme teórica}
\end{Highlighting}
\end{Shaded}

\pandocbounded{\includegraphics[keepaspectratio]{02_generacion_numeros_pseudoaleatorios_files/figure-pdf/histograma_numeros-1.pdf}}

El histograma muestra la distribución de los números pseudoaleatorios
generados. La línea roja representa la densidad uniforme teórica. Si las
barras del histograma están cerca de esta línea, indica que los números
generados están distribuidos uniformemente.

\subsubsection{Análisis de
Independencia}\label{anuxe1lisis-de-independencia}

Estas pruebas verifican si los números en la secuencia son
independientes entre sí, un requisito fundamental para la aleatoriedad.

\paragraph{Gráfico Secuencial y de Dispersión
Retardado}\label{gruxe1fico-secuencial-y-de-dispersiuxf3n-retardado}

Estos gráficos son herramientas visuales para detectar patrones. El
gráfico secuencial muestra los valores en el orden en que fueron
generados, mientras que el de dispersión retardado compara cada valor
con el siguiente

\begin{Shaded}
\begin{Highlighting}[]
\CommentTok{\# Gráfico secuencial y de dispersión retardado}
\FunctionTok{par}\NormalTok{(}\AttributeTok{mfrow =} \FunctionTok{c}\NormalTok{(}\DecValTok{1}\NormalTok{, }\DecValTok{2}\NormalTok{))}
\FunctionTok{plot}\NormalTok{(}\FunctionTok{as.ts}\NormalTok{(numeros\_generados), }\AttributeTok{main =} \StringTok{"Gráfico Secuencial"}\NormalTok{, }\AttributeTok{xlab =} \StringTok{"Índice"}\NormalTok{, }\AttributeTok{ylab =} \StringTok{"Valor"}\NormalTok{)}
\FunctionTok{plot}\NormalTok{(numeros\_generados[}\SpecialCharTok{{-}}\DecValTok{100}\NormalTok{], numeros\_generados[}\SpecialCharTok{{-}}\DecValTok{1}\NormalTok{], }
     \AttributeTok{main =} \StringTok{"Dispersión Retardado (lag 1)"}\NormalTok{, }\AttributeTok{xlab =} \StringTok{"U\_t"}\NormalTok{, }\AttributeTok{ylab =} \StringTok{"U\_t+1"}\NormalTok{)}
\end{Highlighting}
\end{Shaded}

\pandocbounded{\includegraphics[keepaspectratio]{02_generacion_numeros_pseudoaleatorios_files/figure-pdf/graficos_independencia-1.pdf}}

\begin{Shaded}
\begin{Highlighting}[]
\FunctionTok{par}\NormalTok{(}\AttributeTok{mfrow =} \FunctionTok{c}\NormalTok{(}\DecValTok{1}\NormalTok{, }\DecValTok{1}\NormalTok{))}
\end{Highlighting}
\end{Shaded}

El gráfico secuencial muestra la evolución de los números generados a lo
largo del tiempo. Si no hay patrones evidentes, sugiere independencia.
El gráfico de dispersión retardado (lag 1) compara cada número con el
siguiente. Si los puntos están dispersos sin formar patrones claros,
indica independencia entre los números consecutivos.

\paragraph{Prueba de Autocorrelación (Correlograma y Test de
Ljung-Box)}\label{prueba-de-autocorrelaciuxf3n-correlograma-y-test-de-ljung-box}

El correlograma (página 36) muestra las autocorrelaciones para
diferentes retardos (lags). Para una secuencia independiente, estas
autocorrelaciones deberían estar dentro de los límites de confianza
(líneas azules). El test de Ljung-Box es la prueba formal asociada.

\begin{Shaded}
\begin{Highlighting}[]
\CommentTok{\# Correlograma}
\FunctionTok{acf}\NormalTok{(numeros\_generados, }\AttributeTok{main =} \StringTok{"Función de Autocorrelación"}\NormalTok{)}
\end{Highlighting}
\end{Shaded}

\pandocbounded{\includegraphics[keepaspectratio]{02_generacion_numeros_pseudoaleatorios_files/figure-pdf/prueba_autocorrelacion-1.pdf}}

El correlograma muestra las autocorrelaciones de la secuencia de números
generados para diferentes retardos (lags). Si las barras están dentro de
los límites de confianza (líneas azules), sugiere que no hay
autocorrelación significativa, indicando independencia entre los números
generados.

\begin{Shaded}
\begin{Highlighting}[]
\CommentTok{\# Test de Ljung{-}Box}
\FunctionTok{Box.test}\NormalTok{(numeros\_generados, }\AttributeTok{lag =} \DecValTok{10}\NormalTok{, }\AttributeTok{type =} \StringTok{"Ljung{-}Box"}\NormalTok{)}
\end{Highlighting}
\end{Shaded}

\begin{verbatim}

    Box-Ljung test

data:  numeros_generados
X-squared = 11.542, df = 10, p-value = 0.3168
\end{verbatim}

El resultado del test de Ljung-Box proporciona un valor p que indica si
se puede rechazar la hipótesis nula de independencia. Un valor p alto
sugiere que no hay evidencia suficiente para rechazar la hipótesis nula,
indicando independencia entre los números generados.

\subsubsection{Comportamiento Estadístico General y Limitaciones del
Período}\label{comportamiento-estaduxedstico-general-y-limitaciones-del-peruxedodo}

Aunque esta secuencia particular de 100 números pasa las pruebas de
uniformidad e independencia, es crucial recordar la principal debilidad
del método de los cuadrados medios: su corto período y su tendencia a
degenerar. Como se demostró en el documento original, con una semilla
como 1234 y k=4, la secuencia degenera rápidamente a cero. La elección
de una semilla y un valor de k más grandes (123456 y k=6) ha permitido
generar una secuencia corta que parece estadísticamente aceptable. Sin
embargo, para aplicaciones serias que requieren secuencias largas (miles
o millones de números), este generador no es fiable, ya que
eventualmente entrará en un ciclo corto o degenerará.

\subsubsection{Conclusión}\label{conclusiuxf3n}

Para la secuencia corta de 100 valores analizada, el generador de
cuadrados medios con la semilla y k especificados produce resultados que
son consistentes con una muestra uniforme e independiente. Pasa
satisfactoriamente las pruebas de Chi-cuadrado, Kolmogorov-Smirnov,
Ljung-Box y los análisis gráficos. No obstante, este resultado no valida
al generador para un uso general, debido a sus conocidas y graves
deficiencias estructurales, como el corto período y la degeneración, que
lo hacen inadecuado para la mayoría de las aplicaciones de
simulación.ación, que lo hacen inadecuado para la mayoría de las
aplicaciones de simulación.

\subsection{Análisis de la Calidad del generador por el método de los
cuadrados medios sobre Múltiples
Secuencias}\label{anuxe1lisis-de-la-calidad-del-generador-por-el-muxe9todo-de-los-cuadrados-medios-sobre-muxfaltiples-secuencias}

Para obtener una evaluación más robusta de la calidad del generador de
números pseudoaleatorios basado en el método de los cuadrados medios, se
pueden generar múltiples secuencias independientes y aplicar las pruebas
estadísticas a cada una. Esto permite observar la consistencia del
generador a través de diferentes ejecuciones.

Por estos motivos se aplicara la siguiente metodologa para evaluar la
calidad del generador sobre múltiples secuencias:

\begin{longtable}[]{@{}
  >{\raggedright\arraybackslash}p{(\linewidth - 4\tabcolsep) * \real{0.3333}}
  >{\raggedright\arraybackslash}p{(\linewidth - 4\tabcolsep) * \real{0.3333}}
  >{\raggedright\arraybackslash}p{(\linewidth - 4\tabcolsep) * \real{0.3333}}@{}}
\toprule\noalign{}
\begin{minipage}[b]{\linewidth}\raggedright
Paso
\end{minipage} & \begin{minipage}[b]{\linewidth}\raggedright
Descripción
\end{minipage} & \begin{minipage}[b]{\linewidth}\raggedright
Detalles Clave
\end{minipage} \\
\midrule\noalign{}
\endhead
\bottomrule\noalign{}
\endlastfoot
1 & \textbf{Generación de Secuencias} & Se generarán \textbf{100
secuencias} de \textbf{100 números} cada una, utilizando el método de
los cuadrados medios (\texttt{k=6}) con 100 semillas aleatorias
distintas. \\
2 & \textbf{Aplicación de Pruebas} & A cada una de las 100 secuencias se
le aplicarán individualmente las pruebas de \textbf{Chi-cuadrado},
\textbf{Kolmogorov-Smirnov} (para uniformidad) y \textbf{Ljung-Box}
(para independencia). \\
3 & \textbf{Análisis Agregado} & Se recopilarán los \textbf{p-valores}
de todas las pruebas para calcular la \textbf{proporción de rechazos}
(fallos) y se analiza su distribución para evaluar la consistencia del
generador. \\
\end{longtable}

\subsubsection{\texorpdfstring{Implementación en
{R}}{Implementación en R}}\label{implementaciuxf3n-en-r}

Primero, preparamos el entorno y definimos los parámetros para la
simulación.

\begin{Shaded}
\begin{Highlighting}[]
\CommentTok{\# Parámetros de la simulación}
\NormalTok{num\_secuencias }\OtherTok{\textless{}{-}} \DecValTok{100}
\NormalTok{n\_por\_secuencia }\OtherTok{\textless{}{-}} \DecValTok{100}
\NormalTok{k }\OtherTok{\textless{}{-}} \DecValTok{6}
\NormalTok{alpha }\OtherTok{\textless{}{-}} \FloatTok{0.05}

\CommentTok{\# Generar 100 semillas aleatorias de 6 dígitos para asegurar consistencia con k}
\FunctionTok{set.seed}\NormalTok{(}\DecValTok{123}\NormalTok{) }\CommentTok{\# Para la reproducibilidad de las semillas}
\NormalTok{semillas }\OtherTok{\textless{}{-}} \FunctionTok{sample}\NormalTok{(}\DecValTok{100000}\SpecialCharTok{:}\DecValTok{999999}\NormalTok{, num\_secuencias)}

\CommentTok{\# Vectores para almacenar los p{-}valores de cada prueba}
\NormalTok{p\_valores\_chisq }\OtherTok{\textless{}{-}} \FunctionTok{numeric}\NormalTok{(num\_secuencias)}
\NormalTok{p\_valores\_ks }\OtherTok{\textless{}{-}} \FunctionTok{numeric}\NormalTok{(num\_secuencias)}
\NormalTok{p\_valores\_ljungbox }\OtherTok{\textless{}{-}} \FunctionTok{numeric}\NormalTok{(num\_secuencias)}
\end{Highlighting}
\end{Shaded}

Ahora, iteramos sobre las semillas, generamos cada secuencia y aplicamos
las pruebas.

\begin{Shaded}
\begin{Highlighting}[]
\CommentTok{\# Bucle para generar secuencias y aplicar pruebas}
\ControlFlowTok{for}\NormalTok{ (i }\ControlFlowTok{in} \DecValTok{1}\SpecialCharTok{:}\NormalTok{num\_secuencias) \{}
  \CommentTok{\# Generar la secuencia}
\NormalTok{  seq\_actual }\OtherTok{\textless{}{-}}\NormalTok{ simres}\SpecialCharTok{::}\FunctionTok{rvng}\NormalTok{(n\_por\_secuencia, }\AttributeTok{seed =}\NormalTok{ semillas[i], }\AttributeTok{k =}\NormalTok{ k)}
  
  \CommentTok{\# 1. Prueba de Chi{-}cuadrado para uniformidad}
\NormalTok{  test\_chisq }\OtherTok{\textless{}{-}} \FunctionTok{suppressWarnings}\NormalTok{(}
\NormalTok{    simres}\SpecialCharTok{::}\FunctionTok{freq.test}\NormalTok{(seq\_actual, }\AttributeTok{nclass =} \DecValTok{10}\NormalTok{)}
\NormalTok{  )}
\NormalTok{  p\_valores\_chisq[i] }\OtherTok{\textless{}{-}}\NormalTok{ test\_chisq}\SpecialCharTok{$}\NormalTok{p.value}
  
  \CommentTok{\# 2. Prueba de Kolmogorov{-}Smirnov para uniformidad}
  \CommentTok{\# Se suprimen las advertencias sobre empates, que son esperadas.}
\NormalTok{  test\_ks }\OtherTok{\textless{}{-}} \FunctionTok{suppressWarnings}\NormalTok{(}\FunctionTok{ks.test}\NormalTok{(seq\_actual, }\StringTok{"punif"}\NormalTok{))}
\NormalTok{  p\_valores\_ks[i] }\OtherTok{\textless{}{-}}\NormalTok{ test\_ks}\SpecialCharTok{$}\NormalTok{p.value}
  
  \CommentTok{\# 3. Prueba de Ljung{-}Box para independencia}
\NormalTok{  test\_ljungbox }\OtherTok{\textless{}{-}} \FunctionTok{Box.test}\NormalTok{(seq\_actual, }\AttributeTok{lag =} \DecValTok{10}\NormalTok{, }\AttributeTok{type =} \StringTok{"Ljung{-}Box"}\NormalTok{)}
\NormalTok{  p\_valores\_ljungbox[i] }\OtherTok{\textless{}{-}}\NormalTok{ test\_ljungbox}\SpecialCharTok{$}\NormalTok{p.value}
\NormalTok{\}}
\end{Highlighting}
\end{Shaded}

Finalmente, analizamos los resultados agregados de las pruebas.

\subsubsection{Resumen Comparativo de
Resultados}\label{resumen-comparativo-de-resultados}

Con los p-valores de las 100 simulaciones, podemos calcular la
proporción de veces que cada prueba rechazó la hipótesis nula (es decir,
falló en demostrar calidad estadística) a un nivel de significancia de
0.05.

\begin{Shaded}
\begin{Highlighting}[]
\CommentTok{\# Calcular la proporción de rechazos}
\NormalTok{rechazos\_chisq }\OtherTok{\textless{}{-}} \FunctionTok{sum}\NormalTok{(p\_valores\_chisq }\SpecialCharTok{\textless{}}\NormalTok{ alpha) }\SpecialCharTok{/}\NormalTok{ num\_secuencias}
\NormalTok{rechazos\_ks }\OtherTok{\textless{}{-}} \FunctionTok{sum}\NormalTok{(p\_valores\_ks }\SpecialCharTok{\textless{}}\NormalTok{ alpha) }\SpecialCharTok{/}\NormalTok{ num\_secuencias}
\NormalTok{rechazos\_ljungbox }\OtherTok{\textless{}{-}} \FunctionTok{sum}\NormalTok{(p\_valores\_ljungbox }\SpecialCharTok{\textless{}}\NormalTok{ alpha) }\SpecialCharTok{/}\NormalTok{ num\_secuencias}

\CommentTok{\# Crear tabla de resumen}
\NormalTok{resumen }\OtherTok{\textless{}{-}} \FunctionTok{data.frame}\NormalTok{(}
  \AttributeTok{Prueba =} \FunctionTok{c}\NormalTok{(}\StringTok{"Chi{-}cuadrado (Uniformidad)"}\NormalTok{, }\StringTok{"Kolmogorov{-}Smirnov (Uniformidad)"}\NormalTok{, }\StringTok{"Ljung{-}Box (Independencia)"}\NormalTok{),}
  \AttributeTok{Proporcion\_Rechazos =} \FunctionTok{c}\NormalTok{(rechazos\_chisq, rechazos\_ks, rechazos\_ljungbox),}
  \AttributeTok{Proporcion\_Aceptaciones =} \FunctionTok{c}\NormalTok{(}\DecValTok{1} \SpecialCharTok{{-}}\NormalTok{ rechazos\_chisq, }\DecValTok{1} \SpecialCharTok{{-}}\NormalTok{ rechazos\_ks, }\DecValTok{1} \SpecialCharTok{{-}}\NormalTok{ rechazos\_ljungbox)}
\NormalTok{)}

\NormalTok{knitr}\SpecialCharTok{::}\FunctionTok{kable}\NormalTok{(resumen, }
             \AttributeTok{caption =} \StringTok{"Proporción de secuencias que pasan o fallan las pruebas de calidad (alpha = 0.05)."}\NormalTok{,}
             \AttributeTok{col.names =} \FunctionTok{c}\NormalTok{(}\StringTok{"Prueba Estadística"}\NormalTok{, }\StringTok{"Proporción de Fallos (Rechazo H₀)"}\NormalTok{, }\StringTok{"Proporción de Aceptaciones (No Rechazo H₀)"}\NormalTok{))}
\end{Highlighting}
\end{Shaded}

\begin{longtable}[]{@{}
  >{\raggedright\arraybackslash}p{(\linewidth - 4\tabcolsep) * \real{0.3000}}
  >{\raggedleft\arraybackslash}p{(\linewidth - 4\tabcolsep) * \real{0.3091}}
  >{\raggedleft\arraybackslash}p{(\linewidth - 4\tabcolsep) * \real{0.3909}}@{}}
\caption{Proporción de secuencias que pasan o fallan las pruebas de
calidad (alpha = 0.05).}\tabularnewline
\toprule\noalign{}
\begin{minipage}[b]{\linewidth}\raggedright
Prueba Estadística
\end{minipage} & \begin{minipage}[b]{\linewidth}\raggedleft
Proporción de Fallos (Rechazo H₀)
\end{minipage} & \begin{minipage}[b]{\linewidth}\raggedleft
Proporción de Aceptaciones (No Rechazo H₀)
\end{minipage} \\
\midrule\noalign{}
\endfirsthead
\toprule\noalign{}
\begin{minipage}[b]{\linewidth}\raggedright
Prueba Estadística
\end{minipage} & \begin{minipage}[b]{\linewidth}\raggedleft
Proporción de Fallos (Rechazo H₀)
\end{minipage} & \begin{minipage}[b]{\linewidth}\raggedleft
Proporción de Aceptaciones (No Rechazo H₀)
\end{minipage} \\
\midrule\noalign{}
\endhead
\bottomrule\noalign{}
\endlastfoot
Chi-cuadrado (Uniformidad) & 0.14 & 0.86 \\
Kolmogorov-Smirnov (Uniformidad) & 0.13 & 0.87 \\
Ljung-Box (Independencia) & 0.16 & 0.84 \\
\end{longtable}

El resumen muestra la proporción de secuencias que pasaron o fallaron
cada una de las pruebas estadísticas. Idealmente, para un buen generador
de números pseudoaleatorios, se esperaría que la proporción de rechazos
sea cercana al nivel de significancia (0.05 no en este caso), indicando
que las secuencias generadas no son consistentes con las propiedades
esperadas de uniformidad e independencia.

Esto último debido a que los resultados muestran una clara deficiencia
en el generador. Si el generador produjera secuencias verdaderamente
aleatorias, esperaríamos que la proporción de fallos (rechazos de la
hipótesis nula) fuera aproximadamente del 5\% (alpha = 0.05). Sin
embargo, los resultados obtenidos son:

\begin{itemize}
\tightlist
\item
  14\% para la prueba de Chi-cuadrado.
\item
  13\% para la prueba de Kolmogorov-Smirnov.
\item
  16\% para la prueba de Ljung-Box.
\end{itemize}

Estas proporciones son casi tres veces mayores que el nivel de
significancia esperado. Esto indica que el generador falla en producir
secuencias que aparenten ser uniformes e independientes con mucha más
frecuencia de la debida, revelando una inconsistencia sistemática en su
comportamiento.

\subsubsection{Distribución de los
P-valores}\label{distribuciuxf3n-de-los-p-valores}

Además de la proporción de rechazos, es útil visualizar la distribución
de los p-valores obtenidos en las pruebas para evaluar si se comportan
como se esperaría bajo la hipótesis nula.

\begin{Shaded}
\begin{Highlighting}[]
\CommentTok{\# Gráficos de los p{-}valores}
\FunctionTok{par}\NormalTok{(}\AttributeTok{mfrow =} \FunctionTok{c}\NormalTok{(}\DecValTok{1}\NormalTok{, }\DecValTok{3}\NormalTok{), }\AttributeTok{mar =} \FunctionTok{c}\NormalTok{(}\DecValTok{4}\NormalTok{, }\DecValTok{4}\NormalTok{, }\DecValTok{3}\NormalTok{, }\DecValTok{1}\NormalTok{))}
\FunctionTok{hist}\NormalTok{(p\_valores\_chisq, }\AttributeTok{main =} \StringTok{"P{-}valores: Chi{-}cuadrado"}\NormalTok{, }\AttributeTok{xlab =} \StringTok{"p{-}valor"}\NormalTok{, }\AttributeTok{freq =} \ConstantTok{FALSE}\NormalTok{, }\AttributeTok{breaks =} \DecValTok{10}\NormalTok{)}
\FunctionTok{abline}\NormalTok{(}\AttributeTok{h =} \DecValTok{1}\NormalTok{, }\AttributeTok{col =} \StringTok{"red"}\NormalTok{, }\AttributeTok{lty =} \DecValTok{2}\NormalTok{)}
\FunctionTok{hist}\NormalTok{(p\_valores\_ks, }\AttributeTok{main =} \StringTok{"P{-}valores: K{-}S"}\NormalTok{, }\AttributeTok{xlab =} \StringTok{"p{-}valor"}\NormalTok{, }\AttributeTok{freq =} \ConstantTok{FALSE}\NormalTok{, }\AttributeTok{breaks =} \DecValTok{10}\NormalTok{)}
\FunctionTok{abline}\NormalTok{(}\AttributeTok{h =} \DecValTok{1}\NormalTok{, }\AttributeTok{col =} \StringTok{"red"}\NormalTok{, }\AttributeTok{lty =} \DecValTok{2}\NormalTok{)}
\FunctionTok{hist}\NormalTok{(p\_valores\_ljungbox, }\AttributeTok{main =} \StringTok{"P{-}valores: Ljung{-}Box"}\NormalTok{, }\AttributeTok{xlab =} \StringTok{"p{-}valor"}\NormalTok{, }\AttributeTok{freq =} \ConstantTok{FALSE}\NormalTok{, }\AttributeTok{breaks =} \DecValTok{10}\NormalTok{)}
\FunctionTok{abline}\NormalTok{(}\AttributeTok{h =} \DecValTok{1}\NormalTok{, }\AttributeTok{col =} \StringTok{"red"}\NormalTok{, }\AttributeTok{lty =} \DecValTok{2}\NormalTok{)}
\end{Highlighting}
\end{Shaded}

\pandocbounded{\includegraphics[keepaspectratio]{02_generacion_numeros_pseudoaleatorios_files/figure-pdf/distribucion_pvalores-1.pdf}}

\begin{Shaded}
\begin{Highlighting}[]
\FunctionTok{par}\NormalTok{(}\AttributeTok{mfrow =} \FunctionTok{c}\NormalTok{(}\DecValTok{1}\NormalTok{, }\DecValTok{1}\NormalTok{))}
\end{Highlighting}
\end{Shaded}

Los histogramas de los p-valores se desvían notablemente de la
distribución uniforme esperada (representada por la línea roja
discontinua), lo que confirma las deficiencias del generador:

\begin{itemize}
\tightlist
\item
  P-valores de Chi-cuadrado: Se observa una fuerte acumulación de
  p-valores en el extremo izquierdo del histograma (intervalo {[}0.0,
  0.1{]}). Esto indica que un número desproporcionadamente alto de
  secuencias fallaron la prueba de uniformidad, lo cual es consistente
  con la tasa de rechazo del 14\% observada en la tabla.
\item
  P-valores de K-S: Este gráfico muestra una distribución en forma de
  ``U'', con picos tanto en valores muy bajos como muy altos. Esta es
  una señal clásica de un generador defectuoso. No solo produce
  secuencias que se desvían significativamente de la uniformidad
  (p-valores bajos), sino que también genera secuencias que son
  ``demasiado perfectas'' o sospechosamente uniformes (p-valores altos),
  algo que tampoco es esperable en un proceso verdaderamente aleatorio.
\item
  P-valores de Ljung-Box: Al igual que en la prueba de Chi-cuadrado, se
  observa una clara tendencia a generar p-valores bajos, lo que llevó a
  una tasa de rechazo del 16\% y evidencia problemas sistemáticos con la
  independencia de los números generados.
\end{itemize}

\subsubsection{Conclusión Final sobre Estabilidad y
Confiabilidad}\label{conclusiuxf3n-final-sobre-estabilidad-y-confiabilidad}

El análisis sobre 100 secuencias independientes demuestra de manera
concluyente que el método de los cuadrados medios no es un generador de
números pseudoaleatorios confiable ni estable, ni siquiera para
secuencias cortas con semillas aparentemente bien elegidas.

Mientras que el análisis de una sola secuencia puede, por azar, no
revelar problemas, la evaluación repetida muestra un patrón claro de
fallos estadísticos. Las tasas de rechazo para las pruebas de
uniformidad (14\% y 13\%) e independencia (16\%) son casi tres veces
superiores al nivel de significancia esperado del 5\%. Esta
discrepancia, visualmente confirmada por la distribución no uniforme de
los p-valores, indica que el método no genera de forma consistente
secuencias que imiten adecuadamente las propiedades de una muestra
aleatoria.

En definitiva, el método de los cuadrados medios es estadísticamente
deficiente. Su valor es puramente histórico y didáctico. Para cualquier
aplicación práctica, es imperativo utilizar generadores modernos y
robustos, como el Mersenne-Twister, que ha superado rigurosas baterías
de pruebas y es el predeterminado en {R}. que ha superado rigurosas
baterías de pruebas y es el predeterminado en {R}.




\end{document}
